\documentclass{sig-alternate-2013}

\setlength{\paperheight}{11in}
\setlength{\paperwidth}{8.5in}
\usepackage{paralist}
%\usepackage{bm}
\usepackage{algorithm,algorithmic}
%\usepackage{times}
\usepackage{url}
\usepackage{subfigure}
\usepackage{comment}
\usepackage{mathtools}
\usepackage{hyperref}
\usepackage{amsmath}
\usepackage{xspace}

\DeclareMathOperator*{\argmin}{arg\,min}
\DeclarePairedDelimiter{\ceil}{\lceil}{\rceil}


%\usepackage[none]{hyphenat}
%\hyphenchar\font=5

\newtheorem{thm}{Theorem}
\newtheorem{prop}{Proposition}
\newtheorem{lemma}{Lemma}
\newtheorem{cor}[thm]{Corollary}
\newtheorem{assumption}[thm]{Assumption}
\newtheorem{defn}{Definition}[section]

\newfont{\mycrnotice}{ptmr8t at 7pt}
\newfont{\myconfname}{ptmri8t at 7pt}
\let\crnotice\mycrnotice%
\let\confname\myconfname%

\newcommand{\x}{\boldsymbol{x}}
\newcommand{\xb}{\bar{\boldsymbol{x}}}
\newcommand{\y}{\boldsymbol{y}}
\newcommand{\yb}{\bar{\boldsymbol{y}}}
\newcommand{\z}{\boldsymbol{z}}
\newcommand{\zb}{\bar{\boldsymbol{z}}}
\newcommand{\0}{\boldsymbol{0}}
\newcommand{\g}{\boldsymbol{g}}
\newcommand{\gb}{\bar{\boldsymbol{g}}}
\newcommand{\bv}{\boldsymbol{v}}
\newcommand{\bd}{\boldsymbol{d}}
\newcommand{\bb}{\boldsymbol{b}}
\newcommand{\bg}{\boldsymbol{g}}
\newcommand{\Row}{\mathcal{R}}
\newcommand{\lambdamin}{\lambda_{min}}
\newcommand{\lambdamax}{\lambda_{max}}

\newcommand{\Ht}{\tilde{H}}
\newcommand{\Lt}{\tilde{L}}
\newcommand{\Ut}{\tilde{U}}
\newcommand{\Dt}{\tilde{D}}
\newcommand{\Lambdat}{\tilde{\Lambda}}

\newcommand{\eps}{\epsilon}

\newcommand{\Hb}{\bar{H}}
\newcommand{\hb}{\bar{h}}
\newcommand{\bhb}{\boldsymbol{\bar{h}}}
\newcommand{\Lb}{\bar{L}}
\newcommand{\Ub}{\bar{U}}

\newcommand{\Hv}{\vec{H}}
\newcommand{\hv}{\vec{h}}
\newcommand{\bhv}{\boldsymbol{\vec{h}}}
\newcommand{\Lv}{\vec{L}}
\newcommand{\Uv}{\vec{U}}

\newcommand{\Hbt}{\tilde{\Hb}}
\newcommand{\Lbt}{\tilde{\Lb}}
\newcommand{\Ubt}{\tilde{\Ub}}

\newcommand{\Hvt}{\tilde{\Hv}}
\newcommand{\Lvt}{\tilde{\Lv}}
\newcommand{\Uvt}{\tilde{\Uv}}
\newcommand{\bhvt}{\boldsymbol{\tilde{\vec{h}}}}

\newcommand{\I}{\mathcal{I}}
\newcommand{\Z}{\mathcal{Z}}
\newcommand{\N}{\mathcal{N}}
%\newcommand{\R}{\mathcal{R}}
\newcommand{\T}{\mathcal{T}}
\newcommand{\prox}{\mathbf{prox}}
\newcommand{\proj}{\mathbf{proj}}
\newcommand{\brho}{\boldsymbol{\rho}}
\newcommand{\rhob}{\bar{\rho}}
\newcommand{\brhob}{\bar{\brho}}
\newcommand{\diag}{\text{diag}}

\newcommand{\Dx}{\Delta^{P}\x}
\newcommand{\Dxb}{\Delta^{P}\xb}
\newcommand{\Eps}{\mathcal{E}}
\newcommand{\Epsb}{\bar{\Eps}}
\newcommand{\Set}{\mathcal{S}}
\newcommand{\Xb}{\bar{\mathcal{X}}}
\newcommand{\Orth}{\bar{\mathcal{O}}}

\newcommand{\fix}{\marginpar{FIX}}
\newcommand{\new}{\marginpar{NEW}}
\newcommand{\D}{\mathcal{D}}
\newcommand{\R}{\mathbb{R}}
\newcommand{\V}{{\mathcal{V}}}
\newcommand{\A}{{\mathcal{A}}}
\newcommand{\bzero}{{\boldsymbol{0}}}
\newcommand{\bm}{{\boldsymbol{m}}}
\newcommand{\bmm}{{\boldsymbol{m}}}
\newcommand{\bc}{{\boldsymbol{c}}}
\newcommand{\by}{{\boldsymbol{y}}}
\newcommand{\be}{{\boldsymbol{e}}}
\newcommand{\ba}{{\boldsymbol{a}}}
\newcommand{\bw}{{\boldsymbol{w}}}
\newcommand{\bx}{{\boldsymbol{x}}}
\newcommand{\bt}{{\boldsymbol{t}}}
\newcommand{\bxt}{{\boldsymbol{x}^t}}
\newcommand{\bz}{{\boldsymbol{z}}}
\newcommand{\bu}{{\boldsymbol{u}}}
\newcommand{\AL}{\boldsymbol{\alpha}}
\newcommand{\BL}{\boldsymbol{\beta}}
\newcommand{\bone}{{\boldsymbol{1}}}
\newcommand{\bp}{{\boldsymbol{p}}}

%\newcolumntype{C}[1]{>{\centering\arraybackslash$}p{#1}<{$}}
\renewcommand{\baselinestretch}{0.97}
\def\diabetes{{\sf diabetes}\xspace}
\def\wine{{\sf wine}\xspace}
\def\german{{\sf german}\xspace }
\def\letter{{\sf letter}\xspace }
\def\pendigit{{\sf pendigit}\xspace }
\def\vehicle{{\sf vehicle}\xspace}
\def\census{{\sf census}\xspace}
\def\kddcup{{\sf kddcup99}\xspace}
\def\shuttle{{\sf shuttle}\xspace}
\def\shuttle{{\sf MNIST2M}\xspace}
\def\covtype{{\sf covtype}\xspace}
\def\cifar{{\sf cifar}\xspace}
\def\liblinear{{\sf LIBLINEAR}\xspace }
\def\LIBSVM{{\sf LIBSVM}\xspace}
\def\SVMLIGHT{{\sf SVMlight}\xspace}
\def\MNIST{{\sf mnist8m}\xspace}
\def\ijcnn{{\sf ijcnn1}\xspace}
\def\covtype{{\sf covtype}\xspace}
\def\webspam{{\sf webspam}\xspace}
\def\cadata{{\sf cadata}\xspace}
\def\MSD{{\sf MSD}\xspace}
\def\cpusmall{{\sf cpusmall}\xspace}
\def\Twitter{{\sf Twitter}\xspace}
\def\CLRKA{{MEKA}\xspace}
\def\DBCD{{PBM}\xspace}

\def\DASGD{{DASGD}\xspace}
\def\Hogwild{{HogWild}\xspace}
\def\PASSCODE{{PASSCoDe}\xspace}

\begin{document}
\title{Scaling Up Asynchronous Stochastic Gradient Descent in Multi-core Machines}

%\numberofauthors{4}
 \author{}


\maketitle
\begin{abstract}
%\boldmath
The abstract goes here.
\end{abstract}
% IEEEtran.cls defaults to using nonbold math in the Abstract.
% This preserves the distinction between vectors and scalars. However,
% if the conference you are submitting to favors bold math in the abstract,
% then you can use LaTeX's standard command \boldmath at the very start
% of the abstract to achieve this. Many IEEE journals/conferences frown on
% math in the abstract anyway.

% no keywords




% For peer review papers, you can put extra information on the cover
% page as needed:
% \ifCLASSOPTIONpeerreview
% \begin{center} \bfseries EDICS Category: 3-BBND \end{center}
% \fi
%
% For peerreview papers, this IEEEtran command inserts a page break and
% creates the second title. It will be ignored for other modes.
%\IEEEpeerreviewmaketitle

\section{Introduction}
\label{sec:intro}

\begin{itemize}
  \item (Cho) SGD is important for machine learning. Not easy to parallelize SGD because of the nature of ``sequential updates''. 
  \item (Cho) Hogwild (and other similar algorithms)
  \item Scalability is an issue when scaling it to many cores (give an example). Explain why.  
  \item Describe our approach. 
  \item Why can we address the scalability issue. 
  \item (Huan) Give some experimental results to show the scalability of our algorithm. 
  \item Paper outline. 
  \end{itemize}

\section{Related Work (Cho)}
\label{sec:related}
Review literature for asynchronous SGD, and other asynchronous updates (coordinate descent). 
I will write this. 

\section{Background (Cho)}
\label{sec:background}
\subsection{Memory Models in Multi-core Machines}

\subsection{Asynchronous Stochastic Gradient Descent}
\begin{enumerate}
  \item SGD updates. 
  \item Hogwild algorithm. 
  \item Why it cannot scale up (do we want to explain that Hogwild is better in sparse data but not in dense data)? 
\end{enumerate}

\section{Proposed Algorithm}
\label{sec:proposed}


\section{Implementation Issues}
\label{sec:implement}

\section{Experimental Results}
\label{sec:exp}

Setup, data, machine, \dots

In this section, we evaluate the empirical performance of our proposed
algorithm \DASGD with other asynchronous algorithms on multi-core machines.
We consider ? real datasets: xxx, xxx, and xxx. Detailed information is shown in Table~\ref{tab:datasets}. 

\begin{table*}
  \centering
  \caption{Dataset statistics \label{tab:datasets}}
%\resizebox{8.5cm}{!}{
  \begin{tabular}{|c|r|r|r|r|r|} \hline
    Dataset & \# training samples & \# testing samples & \# features &  \#nonzeros\\ 
    \hline
    RCV1 & ? & ? & ? & ?\\ 
    \hline
  \end{tabular}
%  }
\end{table*}

We compare include the following implementations into comparison: 
\begin{enumerate}
  \item \DASGD: our proposed method. 
  \item (other variants?)
  \item \Hogwild:
  \item \PASSCODE: 
  \end{enumerate}

We run all the experiments on a machine with 40 cores and\dots 

We include the following

\subsection{Performance on real datasets}

Figure: use all 40 cores; maybe only compare our algorithm with HogWild. 
For each data, show (1) training loss or objective function (2) prediction accuracy. 


\subsection{Scaling in Number of Cores}

Figure: total number of updates vs time

Figure: Scalability (x-axis: seconds $\times$ cores, y-axis: objective function)
Use 4, 10, 20, 40 cores. 
(Maybe for our algorithm and for Hogwild). 


\section{Conclusion and Future Work}
\label{sec:conclusion}

%The conclusion goes here.

\newpage
\bibliographystyle{plain}
\bibliography{IEEEabrv,durable}



\end{document}


